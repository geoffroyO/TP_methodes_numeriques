\documentclass[10pt,a4paper]{article}
\usepackage[utf8]{inputenc}
\usepackage[T1]{fontenc}
\usepackage[french]{babel}
\usepackage{fancyhdr}
\usepackage{vmargin}
\usepackage{amsmath}
\pagestyle{fancy}
\usepackage{graphicx}
\usepackage{amssymb}
\usepackage{caption}

\renewcommand{\headrulewidth}{1pt}
\fancyhead[L]{Méthodes Numériques}
\fancyhead[R]{Ensimag 1A}

\renewcommand{\footrulewidth}{1pt}
\fancyfoot[C]{\textbf{page \thepage}} 
\fancyfoot[L]{2019}

%%%% debut macro %%%%
\newenvironment{changemargin}[2]{\begin{list}{}{%
\setlength{\topsep}{0pt}%
\setlength{\leftmargin}{0pt}%
\setlength{\rightmargin}{0pt}%
\setlength{\listparindent}{\parindent}%
\setlength{\itemindent}{\parindent}%
\setlength{\parsep}{0pt plus 1pt}%
\addtolength{\leftmargin}{#1}%
\addtolength{\rightmargin}{#2}%
}\item }{\end{list}}
%%%% fin macro %%%%

\begin{document}
\begin{center}
     \begin{LARGE}
         \textbf{Rendu TP Méthodes Numériques~:}\\
         \textit{Simulation d'écoulement fluide}\\
         \textit{Jacopo Iollo, Geoffroy Oudoumanessah}
     \end{LARGE}
 \end{center}
\section{Résolution de l'équation de transport diffusion \newline }
\textbf{Question 1:} On cherche deux matrices $M$ et $N$ telles que $\forall k \in {0,...,N_t-1}$ on ait le système:

$$N 
\begin{pmatrix} 
\Phi(0, (k+1)dt) \\
\vdots \\
\Phi(ndx, (k+1)dt) \\
\vdots \\
\Phi((N_x-1)dx, (k+1)dt)
\end{pmatrix} 
= M
\begin{pmatrix}
\Phi(0, kdt) \\
\vdots \\
\Phi(ndx, kdt) \\
\vdots \\
\Phi((N_x-1)dw, kdt)
\end{pmatrix}
$$
Pour répondre à cette question on utilise la formule de l'équation différentielle discrétisée et on sépare les termes en $\Phi(., t+dt)$ et $\Phi(., t)$, on obtient alors:

\begin{multline*}
\Phi(x, t+dt) - \kappa \frac{dt}{dx^2}(\Phi(x+dx, t+dt) - 2\Phi(x, t+dt) + \Phi(x-dx, t+dt)\\ =\Phi(x, t) -c(x)\frac{dt}{2dx}(\Phi(x+dx, t) - \Phi(x-dx,t)) + c(x)^2\frac{dt^2}{2dx^2}(\Phi(x+dx,t)-2\Phi(x,t)+\Phi(x-dx,t))
\end{multline*}

Puis finalement, en arrangeant les termes on trouve la forme suivante:
\begin{multline*}
-\kappa \frac{dt}{dx^2}\Phi(x-dx,t+dt)+(1+2\kappa\frac{dt}{dt^2})\Phi(x,t+dt) -\kappa\frac{dt}{dx^2}\Phi(x+dx,t+dt)\\
=(c(x)\frac{dt}{2dx}+c(x)^2\frac{dt^2}{2dx^2})\Phi(x-dx,t)+(1-2c(x)^2\frac{dt^2}{2dx^2})\Phi(x,t)+(-c(x)\frac{dt}{2dx}+c(x)^2\frac{dt^2}{2dx^2})\Phi(x+dx,t)
\end{multline*}

Enfin, en utilisant la périodicité spatiale de $\Phi$ on en déduit aisément les matrices:
$$
N=\begin{pmatrix}
1+2\kappa\frac{dt}{dx^2} & -\kappa\frac{dt}{dx^2} & 0 & \ldots  & -\kappa\frac{dt}{dx^2} \\
-\kappa\frac{dt}{dx^2} & 1+2\kappa\frac{dt}{dx^2}  & -\kappa\frac{dt}{dx^2} & \ldots & \vdots \\
0 & -\kappa\frac{dt}{dx^2} & \ddots & \ddots & 0 \\
\vdots & \ddots & \ddots & \ddots & -\kappa\frac{dt}{dx^2} \\
-\kappa\frac{dt}{dx^2} & \ldots & 0 & -\kappa\frac{dt}{dx^2} & 1+2\kappa\frac{dt}{dx^2}  
\end{pmatrix}
$$


$$
M=\begin{pmatrix}
a_1 & b_1 & 0 & \ldots  & c_1\\
c_2 & a_2  & b_2 & \ldots & \vdots \\
0 & c_3 & \ddots & \ddots & 0 \\
\vdots & \ddots & \ddots & \ddots & b_{N_{x}-1} \\
b_{N_x}& \ldots & 0 & c_{N_x} & a_{N_x}  
\end{pmatrix}
$$
avec, $\forall k \in {0,...N_x-1}$
\begin{multline*}
a_k = 1-2c(kdx)^2\frac{dt^2}{2dx^2} \\
b_k = c(kdx)\frac{dt}{2dx}(c(kdx)\frac{dt}{dx}-1) \\
c_k = c(kdx)\frac{dt}{2dx}(c(kdx)\frac{dt}{dx}+1) \\
\end{multline*}
\\ \\
\textbf{Question 2:} On a bien N qui est symétrique, pour vérifier qu'elle est définie positive, on va poser la matrice de Jordan: $J=\begin{pmatrix}
0 & 1 & 0 & \ldots & 0 \\
0 & 0 & 1 & \ldots & 0 \\
\vdots & \vdots & \vdots & \ddots & \vdots \\
0 & \ldots & \ldots & \ldots & 1 \\
1 & 0 & 0 & \ldots & 0 
\end{pmatrix}$. On montre que les valeurs propres de la matrice de Jordan sont les racines $N_t^{ième}$ de l'unité. On trouve par ailleurs que $N$ est la matrice issue de l'évaluation d'un certain polynôme par $J$. En posant ce polynôme $Q$, on trouve alors:
$$
Q(X)= 1+2\kappa\frac{dt}{dx^2} -\kappa\frac{dt}{dx^2}(X+X^{N_{t}-1})
$$
On a aussi $J$ diagonalisable $\exists P \in GL_n(\mathbb{C})$ telle que $J=P^{-1}DP$ avec $D=Diag((\omega_k)_{0\leq k \leq N_t} )$, ainsi on obtient alors $Q(J)=P^{-1}Q(D)P$ par bloc on obtient que les valeurs propres de la matrice $N$ sont les évaluations de $Q$ en $(\omega_{k})_{0\leq k \leq N_t}$. Montrons que $\forall k \in {0,...,N_{t-1}}$ on a $Q(\omega_k) > 0$. En effet on a bien $\omega_k + \omega_k^{N_t-1}=e^{\frac{2ik\pi}{N_t}}+ e^{\frac{2ik\pi(N_t-1)}{N_t}}= 2cos(\frac{2k\pi}{N_t})\in \mathbb{R}$ de plus $1+2\kappa\frac{dt}{dx^2} -\kappa\frac{dt}{dx^2}(2cos(\frac{2k\pi}{N_t})) > 0$, $N$ est alors bien symétrique définie positive. \\
Cela justifie donc bien l'utilisation d'une méthode de type Cholesky car on peut écrire, avec $T \in T_n(\mathbb{R})$:
$$T^tT\begin{pmatrix} \Phi(0, (k+1)dt) \\
\vdots \\
\Phi(ndx, (k+1)dt) \\
\vdots \\
\Phi((N_x-1)dx, (k+1)dt)
\end{pmatrix} 
=M
\begin{pmatrix}
\Phi(0, kdt) \\
\vdots \\
\Phi(ndx, kdt) \\
\vdots \\
\Phi((N_x-1)dw, kdt)
\end{pmatrix}
$$
On résout alors: $T^tY=M
\begin{pmatrix}
\Phi(0, kdt) \\
\vdots \\
\Phi(ndx, kdt) \\
\vdots \\
\Phi((N_x-1)dw, kdt)
\end{pmatrix}$ puis $T \begin{pmatrix}\Phi(0, (k+1)dt) \\
\vdots \\
\Phi(ndx, (k+1)dt) \\
\vdots \\
\Phi((N_x-1)dx, (k+1)dt)
\end{pmatrix} = Y$
\\ \\
\textbf{Question 3:} On code les fonctions $\it{up\_sweep\_cholesky}$ et $\it{down\_sweep\_cholesky}$ avec l'algorithme de Cholesky vu en cours. Enfin on utilise l'algorithme de résolution vu précédemment pour coder la fonction $\it{my\_cholesky}$.
\\ \\
\textbf{Question 4:} On initialise d'abord les matrices $M$, $N$, $phi_0$ et la fonction de convection. Enfin on applique $N_t$ fois la fonction $\it{my\_cholesky}$ pour trouver $\Phi$ au temps final on obtient, pour els différentes valeur de $\kappa$ testées:

\begin{center}
\includegraphics[scale=0.35]{I)4)k=10-4.png}
\captionof{figure}{$\kappa = 10^{-4}$}
\label{graphe 1:}
\end{center}

\begin{center}
\includegraphics[scale=0.35]{I)4)k=10-3.png}
\captionof{figure}{$\kappa = 10^{-3}$}
\label{graphe 2:}
\end{center}

\begin{center}
\includegraphics[scale=0.35]{I)4)k=10-2.png}
\captionof{figure}{$\kappa = 10^{-2}$}
\label{graphe 3:}
\end{center}

\textbf{Question 5:} On code assez simplement la fonction $\it{solveur\_1D}$ avec la fonction $\it{umfpack}$ qui résous un système de la forme $Ax=b$. Enfin on écrit une fonction  $\it{solveur\_2D}$ qui utilise la méthode de splitting présentée  et on plot les résultats. On obtient alors:
\begin{center}
\includegraphics[scale=0.3]{test2D.png}
\captionof{figure}{$Solveur 2D$}
\label{graphe 4}
\end{center}
\section{Résolution du problème de Poisson}
\textbf{Question 6:}
On a, $\forall n,m>1$ et pour $\mathbf{k_x}=\frac{2i\pi q}{L_x}$ et $\mathbf{k_y}=\frac{2i\pi p}{L_y}$:
$S_{nm}(f)=\overset{n}{\underset{p=-n }{\sum}}\overset{m}{\underset{q=-m }{\sum}}\hat{f}_{pq}e^{\mathbf{k_y}y}e^{\mathbf{k_x}x} $ et $S_{nm}(\Psi)=\overset{n}{\underset{p=-n }{\sum}}\overset{m}{\underset{q=-m }{\sum}}\hat{\Psi}_{pq}e^{\mathbf{k_y}y}e^{\mathbf{k_x}x}$. De plus, on peut écrire:

$$ \frac{\partial^2 S_{nm}(\Psi)}{\partial x^2}(x,y) + \frac{\partial^2 S_{nm}(\Psi)}{\partial y^2}= S_{nm}(f)(x,y)$$
D'où, en passant faisant la somme des dérivées:
$$\overset{n}{\underset{p=-n }{\sum}}\overset{m}{\underset{q=-m }{\sum}}\hat{\Psi}_{pq}\mathbf{k_x}^2e^{\mathbf{k_y}y}e^{\mathbf{k_x}x}  +\overset{n}{\underset{p=-n }{\sum}}\overset{m}{\underset{q=-m }{\sum}}\hat{\Psi}_{pq}\mathbf{k_y}^2e^{\mathbf{k_y}y}e^{\mathbf{k_x}x} =\overset{n}{\underset{p=-n }{\sum}}\overset{m}{\underset{q=-m }{\sum}}\hat{f}_{pq}e^{\mathbf{k_y}y}e^{\mathbf{k_x}x} $$

Puis finalement, en identifiants les coefficients, on trouve alors:
$$\hat{\Psi}_{pq}=\frac{\hat{f}_{pq}}{\mathbf{k_x}^2+\mathbf{k_y}^2}$$
\\ \\
\textbf{Question 7:} Pour cette question, on écrit la fonction $\it{fftfreq}$ avec la formule donnée.
\\ \\
\textbf{Question 8:} Il suffit de coder la fonction $\it{poisson\_2d}$ en suivant l'algorithme donnée et en faisant attention au cas où $\mathbf{k_x}$ et $\mathbf{k_y}$ sont simultanément nuls.
\\ \\
\textbf{Question 9:} Soit $\alpha \in \mathbb{R}$ et on pose $\Psi_{\alpha}(x,y)=\alpha sin(2\pi x) sin(2 \pi y)$, on a:
$$\Delta \Psi_{\alpha}(x,y)= -8\pi^2\alpha sin(2\pi x) sin(2 \pi y)=sin(2\pi x) sin(2 \pi y)$$
D'où $\alpha = \frac{-1}{8 \pi^2}$ puis $\Psi_{\alpha}(x,y)=\frac{-1}{8 \pi^2} sin(2\pi x) sin(2 \pi y)$, on peut finalement coder la fonction $\textit{solution\_field}$
\\ \\
\textbf{Question 10:} On obtient un test réussi avec une erreur de : $8.6736173799e^{-18}$, on a donc une précision de l'ordre de $10^{-18}$ sur nos calculs qui sont alors très précis.

\begin{center}
\includegraphics[scale=0.35]{poisson_error.png}
\captionof{figure}{Résultats du Test}
\label{graphe 4:}
\end{center}

\textbf{Question 11:} On trouves les coefficients de Fourier de $\hat{\mathbf{u}}_x$ et $\hat{\mathbf{u}}_y$ avec le même raisonnement que lors de la question 8:

$$\Delta S_{nm}(\mathbf{u}_x)(x,y)=-\frac{\partial S_{nm}(\omega)}{\partial y}(x,y)$$
$$\overset{n}{\underset{p=-n }{\sum}}\overset{m}{\underset{q=-m }{\sum}}\hat{\mathbf{u}_x}_{pq}\mathbf{k_x}^2e^{\mathbf{k_y}y}e^{\mathbf{k_x}x}  +\overset{n}{\underset{p=-n }{\sum}}\overset{m}{\underset{q=-m }{\sum}}\hat{\mathbf{u}_x}_{pq}\mathbf{k_y}^2e^{\mathbf{k_y}y}e^{\mathbf{k_x}x} =-\overset{n}{\underset{p=-n }{\sum}}\overset{m}{\underset{q=-m }{\sum}}\hat{\omega}_{pq}\mathbf{k_y}e^{\mathbf{k_y}y}e^{\mathbf{k_x}x} $$

$$\Delta S_{nm}(\mathbf{u}_y)(x,y)=\frac{\partial S_{nm}(\omega)}{\partial x}(x,y)$$
$$\overset{n}{\underset{p=-n }{\sum}}\overset{m}{\underset{q=-m }{\sum}}\hat{\mathbf{u}_y}_{pq}\mathbf{k_x}^2e^{\mathbf{k_y}y}e^{\mathbf{k_x}x}  +\overset{n}{\underset{p=-n }{\sum}}\overset{m}{\underset{q=-m }{\sum}}\hat{\mathbf{u}_y}_{pq}\mathbf{k_y}^2e^{\mathbf{k_y}y}e^{\mathbf{k_x}x} =\overset{n}{\underset{p=-n }{\sum}}\overset{m}{\underset{q=-m }{\sum}}\hat{\omega}_{pq}\mathbf{k_x}e^{\mathbf{k_y}y}e^{\mathbf{k_x}x} $$
\\ \\
Finalement on a les relations suivantes:

$$\hat{\mathbf{u}_x}_{pq}=\frac{-\hat{\omega}_{pq}}{\mathbf{k_x}^2+\mathbf{k_y}^2} $$
$$\hat{\mathbf{u}_x}_{pq}=\frac{\hat{\omega}_{pq}}{\mathbf{k_x}^2+\mathbf{k_y}^2} $$

On code ainsi, dans la fonction $\textit{poisson\_curl\_2d}$ en calculant les $\hat{\mathbf{u}_x}_{pq}$ et $\hat{\mathbf{u}_x}_{pq}$ puis en passant à la transformée inverse de Fourier on trouve $\mathbf{u}_x$ et $\mathbf{u}_y$. Pour le test donné, on obtient les résultats suivants:


\begin{center}
\includegraphics[scale=0.3]{poisson_curlx.png}
\captionof{figure}{Résultats du Test pour $\mathbf{u}_x$}
\label{graphe 4:}
\end{center}

\begin{center}
\includegraphics[scale=0.3]{poisson_curly.png}
\captionof{figure}{Résultats du Test pour $\mathbf{u}_y$}
\label{graphe 4:}
\end{center}

On constate que les erreurs commises sont de l'ordre que celles commises précédemment lors de la question 10, ce qui semble normal puisque l'on utilise la même méthode pour la résolution.

\section{Simulation numérique}

\textbf{Question 12:} On a $\omega^0(x,y) = \frac{\partial \mathbf{u}_y^0}{\partial x} - \frac{\partial \mathbf{u}_x^0}{\partial y}$. De plus, $\frac{\partial \mathbf{u}_y^0}{\partial x}=2 \pi \delta cos(2\pi x)$ et   
\begin{equation*} 
\frac{\partial\mathbf{u}_x^0}{\partial y} =
     \begin{cases}
        \frac{\rho}{ch^2(\rho(y-0.25))} & \text{si $\rho \leq 0.5$} \\
        \frac{-\rho}{ch^2(\rho(0.75-y))}  & \text{si $\rho > 0.5$}
     \end{cases}
\end{equation*}





















\end{document}